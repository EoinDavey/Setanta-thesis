\chapter{Technical Problem Description}
\section{\tsPEG{}}
\tsPEG{} must be designed to meet the following requirements:

 \begin{itemize}
     \item \tsPEG{} must be a fully functional TypeScript parser generator. It should allow the user to specify a PEG grammar in some input format, and output a fully correct parser for that grammar.
 
     \item \tsPEG{} should also be written in TypeScript, to allow it to self-host, and bootstrap. Hence \tsPEG{} will run on the NodeJS JavaScript runtime.
 
     \item The outputted parser must correctly match precisely the input grammar, and report any syntax errors encountered along the way.
 
     \item The parser must utilise the TypeScript type system to the fullest extent, making use of discriminated union types and enum types to create a strongly typed AST for the input grammar.
 
     \item The \tsPEG{} input format should be expressive enough to allow users to specify complex and powerful grammars easily. It should support the standard set of PEG operators.
 \end{itemize}

\section{\Setanta{}}

The technical requirements of \Setanta{} are:
\begin{itemize}
    \item \Setanta{} should be fully usable with no knowledge of English, only Irish can be relied on.
    \item \Setanta{} is required to be executable in the browser as well as locally.
    \item The syntax and semantics of \Setanta{} must be directly influenced by the linguistic and cultural properties of the Irish language, a simple re-skin of an existing language with Irish keywords is not sufficient.
    \item Standard features of modern PLs must be present. The domain of \Setanta{} is education, it must not be esoteric, learning \Setanta{} should teach fundamental programming concepts.
    \item When the opportunity presents itself \Setanta{} should make programming in it as accessible as possible. This includes using simplified vocabulary compared to mainstream languages, as well as not requiring the use of fadas (diacritics like áéíóú), as not all users can type these characters.
\end{itemize}

Outside of these technical requirements, the design of \Setanta{} is largely free for me to decide.

\section{\trys{}}

The learning environment at \trys{} also has some technical requirements. \trys{} exists as a portal to using and learning \Setanta{}. Therefore it must satisfy the following conditions:
\begin{itemize}
    \item \trys{} must enable visitors to the site to write and execute \Setanta{} code.
    \item \trys{} must be clearly laid out and simple to use.
    \item Some method of saving code and sharing it with friends needs to be included.
    \item As with \Setanta{}, \trys{} must not require any knowledge of English, only Irish can be used.
    \item Some sort of graphical display must be included, and should have an API that is exposed to the \Setanta{} runtime.
\end{itemize}
