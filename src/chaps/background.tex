\chapter{Background}

\section{Executing Code on the Browser}

The only PL that standard web browsers (\emph{Google Chrome, Mozilla Firefox, Safari, Edge, Internet Explore etc.}) can execute is JavaScript. There is a movement to bring other PLs to the browser via a technology called WebAssembly, but this has not fully come to fruition yet.  
In addition to this restriction on PLs, each browser instance limits executing code to a single, non-blocking thread.

\subsection{Task Queue APIs}
To overcome the non-blocking limitation of the execution thread the JavaScript environment exposes an API to a simple task queue, allowing users to pass in \textbf{callbacks}, which are functions that will be executed at some later time. Figure \ref{blockingcode} contains some simple code that would work if the browser thread allowed blocking.

\begin{figure}
    \caption{Code if thread can be blocked.}
    \label{blockingcode}
    \begin{lstlisting}[language=javascript]
// Would work in an ideal world
function sleepFor100ms() {
    print('Starting sleep')
    sleep(100)
    print('After sleep')
}
    \end{lstlisting}
\end{figure}

Ideally this would print "Starting Sleep", wait 100ms, then print "After Sleep". However, sleeping would be a blocking action on the thread, so this is not allowed. In JavaScript we must use the \texttt{setTimeout} function, which takes a callback function, and a time to wait before executing it. Figure \ref{setTimeoutCode} shows code that uses the \texttt{setTimeout} API to add a callback to the task queue.

\begin{figure}
    \caption{Using \texttt{setTimeout}}
    \label{setTimeoutCode}
    \begin{lstlisting}[language=javascript]
// The real way to achieve this
function sleepFor100ms() {
    print('Starting sleep')
    setTimeout(() => print('After sleep'), 100)
}
    \end{lstlisting}
\end{figure}

Similarly JavaScript uses an API to overcome the single-thread limitation. The developer is given access to certain functions that can use an extra thread for specific tasks, e.g. Fetching web content from a different website. These APIs also use callback functions to specify what should be done with the result of the functions.

This reliance on callback functions leads to something that the JavaScript community refers to as \textbf{Callback Hell}.

\subsection{async / await and Promises}

In an attempt to overcome callback hell, recent versions of JavaScript have introduced a concept known as "async / await".\cite{es8spec} This feature introduces what is effectively a syntactic sugar over a concept introduced previously, known as \textbf{Promises}.

Promises are a reference to a value that has not yet been computed, but will at some point be computed (resolved). These references are objects that can be manipulated just like any other object.

Mozilla describe Promises as
\begin{quote}
A Promise is a proxy for a value not necessarily known when the promise is created. It allows you to associate handlers with an asynchronous action's eventual success value or failure reason. This lets asynchronous methods return values like synchronous methods: instead of immediately returning the final value, the asynchronous method returns a promise to supply the value at some point in the future.
    \cite{mozillapromises}
\end{quote}

\section{TypeScript}
TypeScript is an open source PL being developed by Microsoft, TypeScript is the primary PL that is used in the implementation of this project. Microsoft describes it as

\begin{quote}
TypeScript is a typed superset of JavaScript that compiles to plain JavaScript.\cite{microsoftts}
\end{quote}

TypeScript allows developers to write JavaScript programs with all the benefits of compile time static type checking. As TypeScript is compiled to JavaScript it allows the user to use features of JavaScript that might not be widely supported yet, as it will compile them down to simpler features. Unfortunately as TypeScript is still compiled to JavaScript, we are left with the same limitations as mentioned before.

\section{PEG and CFG Grammars}

Formal grammars are sets of rules for re-writing strings, first formally researched and defined by Noam Chomsky, 1956\cite{chomskypaper}. In recent times they find frequent use in specifying the syntactic grammar of various PLs.

The formal grammar family that has found the most usage is that of CFGs, Context Free Grammars. PEG Grammars are a new type of grammar that has come into prevalent usage recently, they are very similar to CFG grammars but are defined in such a way that they cannot be ambiguous. It is conjectured that PEGs are more powerful than CFGs\cite{pegconjecture}

\section{Parser Generators}

A parser generator is a program that, given a description of a grammar in some formal syntax, outputs a parser for that grammar. Parser generators have existed since the earliest days of computer programming. One of the earliest major parser generators, YACC, was released in 1975 \cite{yacc}, and was based on LR parsing.

Several parser generators are available for JavaScript, including the excellent \emph{pegjs} generator. However, no such parser generator was available for TypeScript, this is what prompted the creation of \tsPEG{}.

\section{VSO vs SVO languages}

In linguistics there is a concept known as word order, which is studying the order of how the different categories of words appear in sentences. One order that is studied is called the "constituent word order". The constituent word order is the order that a verb (V), object (O), and subject (S) appear in a sentence\cite{wordorder}. English is an SVO language, meaning that the subject comes first, then the verb, then the object. For example
\begin{quote}
    The man drove his car
\end{quote}
In this sentence "the man" is the subject, "drove" is the verb, and "his car" is the object, and they appear in the order SVO.

Irish is a VSO language, meaning the verb comes first, then the subject, then the object. The same above sentence in Irish would be
\begin{quote}
    Thiomáin an fear a charr
\end{quote}
In this case "Thiomáin" (drove) is the verb, "an fear" (the man) is the subject, and "a charr" (his car) is the object.

