\chapter{Conclusion}

\section{Goals and project reception}
The main goals for the project were to create an original Irish programming language, a learning environment, and a parser generator for TypeScript. In reference to these goals, the project has been fully successful.

At the time of writing, \trys{} has been visited \textbf{over 1400 times by over 200 unique users} and the \Setanta{} REPL has been downloaded over 100 times, these parts of the project haven't even been properly announced yet, only to people I directly have spoken to.
\tsPEG{} has been downloaded well over 700 times and has been maintaining an average of over 20 downloads a week.

The reception of the project has been fantastic, there is a lot of interest in Irish language programming out there and I have received highly positive feedback from many people. I gave a lightning talk about the project at SISTEM 2020, which was hosted in UCD and the support and interest I received was very high.

\section{The value of non-English PLs}

In creating \Setanta{}, we intended for the design process to expose some links between standard PLs and English.
As discussed in the section on the @ operator (\ref{atoperator}), even simple, extremely common features such as the dot (.) operator have subtle links to the English language.
Although the links we found were not groundbreaking by any means, the existence of these links is important. Diversity of thought is hampered when all programming is passed through the lens of the English language.
In this project, we've seen that it’s not just the keywords that form this lens, but the deeper design of the PL too.

\Setanta{} wasn't created to be a huge technical breakthrough, in fact, as an educational language, we were aiming for familiarity.
\Setanta{} was created to break the pattern of English PLs, and reveal the limitations that had been placed upon us by them.

While I was working on this project, a student at Carnegie Mellon had a similar idea and created the worlds first classical Chinese PL\cite{chinesepl}. This shows that there is a general interest in non-English PLs in the world.

\section{The future of the project}

This project has been a passion project for me, I have put in far beyond 5 credits of work into this project because of that, and I fully intend to continue its development after the project has been submitted. I plan on a full release of version 1.0 of \Setanta{} soon, after I complete the full documentation.
I hope that \Setanta{} finds use in CS education, there are many schools and institutions around the country that I hope will have some interest in it. I have met many people who have shown great enthusiasm for the project, I plan on opening the project up to the open-source community, hopefully including someone with much better Irish than me.
