\documentclass[11pt]{extarticle}
\usepackage[utf8]{inputenc}
\usepackage[T1]{fontenc}
\usepackage[a4paper, margin=25mm]{geometry}
\usepackage{fancyhdr}
\usepackage{textcomp}
\usepackage{amsmath}
\usepackage{hyperref}
\usepackage{xcolor}
\usepackage{graphicx}
\usepackage{listings}
\usepackage{float}
\floatstyle{boxed} 
\restylefloat{figure}
\graphicspath{ {src/} }
\newcommand{\Setanta}{\emph{Setanta}}
\newcommand{\tsPEG}{\emph{tsPEG}}
\newcommand{\trys}{\href{https://try-setanta.ie}{try-setanta.ie}}
\renewcommand{\baselinestretch}{1.5}
\hypersetup{
    colorlinks=true,
    urlcolor=blue,
}
\lstdefinelanguage{setanta}
{
    morekeywords={
        le, idir, creatlach, toradh
    },
    morestring=[d]'
}
\lstset
{
    numbers=left,
    basicstyle=\footnotesize,
    showstringspaces=false,
    breaklines=true,
    keywordstyle=\color{blue}\bfseries,
    stringstyle=\ttfamily,
    inputencoding=utf8,
    extendedchars=true,
    literate=%
        {á}{{\'a}}1
        {ó}{{\'o}}1
        {é}{{\'e}}1
        {í}{{\'i}}1
        {ú}{{\'u}}1
        {má}{{{\color{blue}\textbf{m\'a}}}}2
        {nó}{{{\color{blue}\textbf{n\'o}}}}2
        {nuair-a}{{{\color{blue}\textbf{nuair-a}}}}7
        {neamhní}{{{\color{purple}\textbf{neamhn\'i}}}}7
        {gníomh}{{{\color{blue}\textbf{gn\'iomh}}}}6
        {fíor}{{{\color{orange}f\'ior}}}4
        {breag}{{{\color{orange}breag}}}5
    }
\pagestyle{fancy}
\fancyhf{}
\rhead{\thepage}
\lhead{Eoin Davey - Final Year Project 2020 - 5 credits}
\def\chapinp#1{\input{src/chaps/#1}}
\begin{document}
    \title{Appendix 2}
    \author{Eoin Davey - 16634926}
    \date{}
    \maketitle
    \thispagestyle{fancy}
The syntax of \Setanta{} is new, but should feel familiar to most people. It has been designed to be simple and approachable. It takes inspiration from C like languages, but has some new ideas of it's own.
The grammar has been designed for unambiguity so no semicolons or similar construct are required.

\Setanta{} programs, like most imperative languages, consist of a sequence of statements. Some important \Setanta{} features are outlined below:
\begin{itemize}
    \item \textbf{Variable declarations}

        In \Setanta{} variables are declared using the \verb|:=| operator, and can be re-assigned using the classic \verb|=| operator. The distinction is to provide a clear lexical difference between variable declaration and reassignment.

            \begin{lstlisting}[language=setanta, frame=single, caption=Variables]
x := 0
x = x + 1
            \end{lstlisting}
    \item \textbf{Conditionals}

        \Setanta{} support the classic conditional execution structure of if, else if, else. This is mostly a direct translation into Irish, as it uses the keyword \lstinline[language=setanta]|má| meaning ``if''. However it should be noted that no bracketing is required around the expression.

            \begin{lstlisting}[language=setanta, frame=single, caption=Setanta conditionals]
má x == 0
    scríobh('Tá x cothrom le 0')
nó má x == 1
    scríobh('Tá x cothrom le 1')
nó
    scríobh('Tá x níos mo ná 1')
            \end{lstlisting}
        \item \textbf{Loops}

            \Setanta{} supports two main types of loops, ``le idir'' loops that allow the user to specify start and ends to the loop, and ``nuair-a'' loops, which are the familiar while loops.

            \begin{lstlisting}[language=setanta, frame=single, caption=Setanta loops]
i := 0
le i idir (0, 10)
    i = i + 1
x := 0
nuair-a x < 10
    x = x + 1
            \end{lstlisting}
        \item \textbf{Functions}

            Functions in \Setanta{} are referred to with the term ``gníomh'' meaning ``action''. They can constructed with the \lstinline[language=setanta]|gníomh| keyword, followed by a name and argument list. The \lstinline[language=setanta]|toradh| keyword can be used to return values from the function.
            \begin{lstlisting}[language=setanta, frame=single, caption=Setanta function example]
gníomh fibonacci(n) {
    má n <= 1
        toradh 1
    toradh fibonacci(n-1) + fibonacci(n-2)
}
            \end{lstlisting}
        \item \textbf{Classes}

            \Setanta{} supports declaring new classes, with methods, and a constructor.  Classes can inherit from other classes using the keyword \emph{ó}
            \begin{lstlisting}[language=setanta, frame=single, caption=Setanta classe example]
creatlach Person ó Animal {
    gníomh nua(name) {
        name@seo = name
    }
    gníomh speak() {
        scríobh('Hi, My name is ' + name@seo)
    }
}
            \end{lstlisting}
        \item \textbf{Literals}

            \Setanta{} supports literals for integers, booleans, null, strings and lists.
            \begin{lstlisting}[language=setanta, frame=single, caption=Setanta literals]
a := 500
b := 'Dia duit domhan'
c := [1,2,3,4, fíor]
d := fíor != breag
c := neamhní
            \end{lstlisting}
\end{itemize}

\end{document}
